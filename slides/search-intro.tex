\begin{frame}
  \frametitle{Remembering the problem}
  \begin{center}
    \includegraphics[width=3.2in]{img/robot_in_kitchen.jpg}

    \tiny{Original image courtesy of Wendy Rogers/Georgia Tech.}
  \end{center}
\end{frame}

\begin{frame}
  \frametitle{State of research}
  \begin{itemize}
  \item With the advent of sampling-based planning methods, robotic
    manipulation of arbitrary objects has become feasible.
  \item The solving of SLAM means that moving through and mapping out
    environments is no longer a barrier to robotic applications.
  \item Active visual search for objects out in the open is nearly a solved
    problem.
  \item Current research on objects occurrence models shows that certain types of
    objects appear together more often than others, and certain types of objects
    are unlikely to appear together.
  \end{itemize}
\end{frame}

\begin{frame}
  \frametitle{Assumptions for the following treatment}
  \begin{itemize}
  \item There are no errors in the visual identification and classification of objects.
  \item The robot can successfully complete any movement and manipulation task,
    but motions take time, and are therefore expensive.
  \item The set of object types in the world is finite.
  \end{itemize}
\end{frame}

\begin{frame}
  \frametitle{Formulation of the problem}
  \begin{center}
    \vspace{-0.13in}
    Set of containers: $\{c_l\}$

    \spL[3-unknown-containers]

    Set of object types: $\{t_i\}$

    \spL[shape-universe-small]

    We want to find an object of type $q$ (the query type).

    \spL[blue-circle]

  \end{center}
\end{frame}

\begin{frame}
  \frametitle{Where is \spM[blue-circle]?}
  \begin{center}
    \spL[3-partially-observed-containers]

    \vspace{0.3in}

    Given that we have observed objects of types $\{t_{o_j}\}$ \\
    in container $c$, what is $\mathrm{P}(q \in c \, | \, \{t_{o_j}\})$ ?
  \end{center}
\end{frame}

